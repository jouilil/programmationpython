%% Generated by Sphinx.
\def\sphinxdocclass{jupyterBook}
\documentclass[letterpaper,10pt,french]{jupyterBook}
\ifdefined\pdfpxdimen
   \let\sphinxpxdimen\pdfpxdimen\else\newdimen\sphinxpxdimen
\fi \sphinxpxdimen=.75bp\relax
\ifdefined\pdfimageresolution
    \pdfimageresolution= \numexpr \dimexpr1in\relax/\sphinxpxdimen\relax
\fi
%% let collapsible pdf bookmarks panel have high depth per default
\PassOptionsToPackage{bookmarksdepth=5}{hyperref}
%% turn off hyperref patch of \index as sphinx.xdy xindy module takes care of
%% suitable \hyperpage mark-up, working around hyperref-xindy incompatibility
\PassOptionsToPackage{hyperindex=false}{hyperref}
%% memoir class requires extra handling
\makeatletter\@ifclassloaded{memoir}
{\ifdefined\memhyperindexfalse\memhyperindexfalse\fi}{}\makeatother

\PassOptionsToPackage{warn}{textcomp}

\catcode`^^^^00a0\active\protected\def^^^^00a0{\leavevmode\nobreak\ }
\usepackage{cmap}
\usepackage{fontspec}
\defaultfontfeatures[\rmfamily,\sffamily,\ttfamily]{}
\usepackage{amsmath,amssymb,amstext}
\usepackage{babel}



\setmainfont{FreeSerif}[
  Extension      = .otf,
  UprightFont    = *,
  ItalicFont     = *Italic,
  BoldFont       = *Bold,
  BoldItalicFont = *BoldItalic
]
\setsansfont{FreeSans}[
  Extension      = .otf,
  UprightFont    = *,
  ItalicFont     = *Oblique,
  BoldFont       = *Bold,
  BoldItalicFont = *BoldOblique,
]
\setmonofont{FreeMono}[
  Extension      = .otf,
  UprightFont    = *,
  ItalicFont     = *Oblique,
  BoldFont       = *Bold,
  BoldItalicFont = *BoldOblique,
]



\usepackage[Sonny]{fncychap}
\ChNameVar{\Large\normalfont\sffamily}
\ChTitleVar{\Large\normalfont\sffamily}
\usepackage[,numfigreset=1,mathnumfig]{sphinx}

\fvset{fontsize=\small}
\usepackage{geometry}


% Include hyperref last.
\usepackage{hyperref}
% Fix anchor placement for figures with captions.
\usepackage{hypcap}% it must be loaded after hyperref.
% Set up styles of URL: it should be placed after hyperref.
\urlstyle{same}


\usepackage{sphinxmessages}



        % Start of preamble defined in sphinx-jupyterbook-latex %
         \usepackage[Latin,Greek]{ucharclasses}
        \usepackage{unicode-math}
        % fixing title of the toc
        \addto\captionsenglish{\renewcommand{\contentsname}{Contents}}
        \hypersetup{
            pdfencoding=auto,
            psdextra
        }
        % End of preamble defined in sphinx-jupyterbook-latex %
        

\title{FSJES Mohammedia, UH2}
\date{févr. 09, 2022}
\release{}
\author{Prof.\@{} Jouilil Youness}
\newcommand{\sphinxlogo}{\vbox{}}
\renewcommand{\releasename}{}
\makeindex
\begin{document}

\ifdefined\shorthandoff
  \ifnum\catcode`\=\string=\active\shorthandoff{=}\fi
  \ifnum\catcode`\"=\active\shorthandoff{"}\fi
\fi

\pagestyle{empty}
\sphinxmaketitle
\pagestyle{plain}
\sphinxtableofcontents
\pagestyle{normal}
\phantomsection\label{\detokenize{intro::doc}}


\begin{sphinxadmonition}{note}{Note:}
\sphinxAtStartPar
Les exercices suivants concernent la série N 1 des travaux dirigés.
\end{sphinxadmonition}


\chapter{Série 1 : Espace vectoriel}
\label{\detokenize{S_xe9rie1:serie-1-espace-vectoriel}}\label{\detokenize{S_xe9rie1::doc}}

\section{Prof. Jouilil Youness}
\label{\detokenize{S_xe9rie1:prof-jouilil-youness}}

\subsection{1. Sous\sphinxhyphen{}espace vectoriel}
\label{\detokenize{S_xe9rie1:sous-espace-vectoriel}}
\begin{sphinxadmonition}{note}{Exercice 1}

\sphinxAtStartPar
\sphinxstylestrong{Montrer que les sous\sphinxhyphen{}ensembles suivants des sous\sphinxhyphen{}espaces vectoriels}
\begin{itemize}
\item {} 
\sphinxAtStartPar
\(A_1\) =\( \left\{ (x,y,z) \in \mathbb{R}^3 / x+y+z = 0 \right\}\)

\item {} 
\sphinxAtStartPar
\(A_2\) = \( \left\{ (x,y,z) \in \mathbb{R}^3 / x-y = 0 \right\}\)

\item {} 
\sphinxAtStartPar
\(A_3\) = \( \left\{ (x,y,z) \in \mathbb{R}^3 / x =3z \right\}\)

\end{itemize}
\end{sphinxadmonition}

\begin{sphinxadmonition}{note}{Réponse de l’exercice 1}
\begin{itemize}
\item {} 
\sphinxAtStartPar
\(A_1\) =\( \left\{ (x,y,z) \in \mathbb{R}^3 / x+y+z = 0 \right\}\)

\end{itemize}

\sphinxAtStartPar
\((0,0,0) ~\in\) \(A_1 ~~ \Rightarrow A_1 \neq  \emptyset\)

\sphinxAtStartPar
Soit :  \(X=(x,y,z) ~ et~ X'= (x',y',z') ~ \in ~A_1\)

\sphinxAtStartPar
On a :

\sphinxAtStartPar
\(X+ X' = (x,y,z) + (x',y',z') = (x+x',y+y',z+z')\)

\sphinxAtStartPar
tel que : \(x+y+z = 0 ~et~ x'+y'+z' = 0 \Rightarrow x+x'+y'+ y+ z +z' = 0 \)

\sphinxAtStartPar
\(X=(x,y,z) ~ et~ X'= (x',y',z') ~ \in ~A_1 \Rightarrow X+ X'~ \in ~A_1\)

\sphinxAtStartPar
\(\lambda X = (\lambda x, \lambda y,\lambda z)\)

\sphinxAtStartPar
tel que :
\(\lambda x+ \lambda y+ \lambda z = 0 \Rightarrow \lambda (x+y+z) = 0\)

\sphinxAtStartPar
Ainsi, \(\lambda X \in ~A_1\)
\begin{itemize}
\item {} 
\sphinxAtStartPar
\(A_2\) = \( \left\{ (x,y,z) \in \mathbb{R}^3 / x-y = 0 \right\}\)

\item {} 
\sphinxAtStartPar
\(A_3\) = \( \left\{ (x,y,z) \in \mathbb{R}^3 / x =3z \right\}\)

\end{itemize}
\end{sphinxadmonition}

\begin{sphinxadmonition}{note}{Exercice 2}

\sphinxAtStartPar
Les sous\sphinxhyphen{}ensembles suivants sont\sphinxhyphen{}ils des sous\sphinxhyphen{}espaces vectoriels ?
\begin{itemize}
\item {} 
\sphinxAtStartPar
\(A_1\) =\( \left\{ (x,y,z) \in \mathbb{R}^3 / x+y-3z = 2 \right\}\)

\item {} 
\sphinxAtStartPar
\(A_2\) =  \(\left\{ (x,y,z) \in \mathbb{R}^3 / x+y > z \right\}\)

\item {} 
\sphinxAtStartPar
\(A_3\) = \( \left\{ (x,y,z) \in \mathbb{R}^3 / xyz = 0 \right\}\)

\item {} 
\sphinxAtStartPar
\(A_4\) = \( \left\{ (x,y,z, t) \in \mathbb{R}^4 / x = 2y= -z=3t \right\}\)

\item {} 
\sphinxAtStartPar
\(A_5\) = \( \left\{ (x,y,z) \in \mathbb{R}^3 / y(x^2+z) = 0 \right\}\)

\end{itemize}
\end{sphinxadmonition}

\begin{sphinxadmonition}{note}{Réponse de l’exercice 2}

\sphinxAtStartPar
Pour
\begin{itemize}
\item {} 
\sphinxAtStartPar
\(A_1\) =\( \left\{ (x,y,z) \in \mathbb{R}^3 / x+y-3z = 2 \right\}\)

\item {} 
\sphinxAtStartPar
\(A_2\) =  \(\left\{ (x,y,z) \in \mathbb{R}^3 / x+y > z \right\}\)

\item {} 
\sphinxAtStartPar
\(A_3\) = \( \left\{ (x,y,z) \in \mathbb{R}^3 / xyz = 0 \right\}\)

\item {} 
\sphinxAtStartPar
\(A_4\) = \( \left\{ (x,y,z, t) \in \mathbb{R}^4 / x = 2y= -z=3t \right\}\)

\item {} 
\sphinxAtStartPar
\(A_5\) = \( \left\{ (x,y,z) \in \mathbb{R}^3 / y(x^2+z) = 0 \right\}\)

\end{itemize}
\end{sphinxadmonition}

\begin{sphinxadmonition}{note}{Exercice 3}
\begin{itemize}
\item {} 
\sphinxAtStartPar
On considère l’ensemble suivant:

\end{itemize}
\begin{equation*}
\begin{split}
A = \left\{ X \in \mathbb{R}^4 ; X= (x+2y,y-z, -x+3z,-z)~~ x,y,z \in \mathbb{R} \right\}
\end{split}
\end{equation*}\begin{enumerate}
\sphinxsetlistlabels{\arabic}{enumi}{enumii}{}{.}%
\item {} 
\sphinxAtStartPar
Montrer que A est un sous espace vectoriel de \(\mathbb{R}^4\)

\item {} 
\sphinxAtStartPar
Donner une base de A

\item {} 
\sphinxAtStartPar
Le vecteur suivant (0,1,\sphinxhyphen{}4,0) appartient\sphinxhyphen{}il à A ? \sphinxstyleemphasis{Justifier votre réponse !}

\end{enumerate}
\end{sphinxadmonition}

\begin{sphinxadmonition}{note}{Réponse de l’exercice 3}

\sphinxAtStartPar
On a :
\begin{equation*}
\begin{split}
A = \left\{ X \in \mathbb{R}^4 ; X= (x+2y,y-z, -x+3z,-z)~~ x,y,z \in \mathbb{R} \right\}
\end{split}
\end{equation*}\begin{enumerate}
\sphinxsetlistlabels{\arabic}{enumi}{enumii}{}{.}%
\item {} 
\sphinxAtStartPar
Montrons que A est un un sous espace vectoriel de \(\mathbb{R}^4\)

\item {} 
\sphinxAtStartPar
Cherchons une base de A.

\item {} 
\sphinxAtStartPar
Le vecteur (0,1,\sphinxhyphen{}4,0) appartient\sphinxhyphen{}il à A ?

\end{enumerate}
\end{sphinxadmonition}


\subsection{2. Combinaison linéaire}
\label{\detokenize{S_xe9rie1:combinaison-lineaire}}
\begin{sphinxadmonition}{note}{Exercice 3}

\sphinxAtStartPar
On considère les vecteurs de \(\mathbb{R}^3\) suivants :
\(
E_1(1,0,1) ; E_2(0,1,1) ; E_3(1,1,0)
\)
\begin{itemize}
\item {} 
\sphinxAtStartPar
Est\sphinxhyphen{}ce que le vecteur \(E\)\((2,3,0)\) est combinaison linéaire des vecteurs \(E_1\), \(E_2\) et \(E_3\) ?

\end{itemize}
\end{sphinxadmonition}

\begin{sphinxadmonition}{note}{Exercice 3}

\sphinxAtStartPar
On considère les ensembles suivants

\sphinxAtStartPar
\( 
E_1 = \left\{  A = \begin{pmatrix}
a& -b\\
b&c \\
\end{pmatrix} \in M(2) ~;~ 
a,b,c \in \mathbb{R} \right\}
\)

\sphinxAtStartPar
\( 
E_2 = \left\{  A = \begin{pmatrix}
a& c\\
b& d\\
\end{pmatrix} \in M(2) ~;~
a,b,c, d \in \mathbb{R} \right\}
\)

\sphinxAtStartPar
\( 
E_3 = \left\{  A = \begin{pmatrix}
a& a & c\\
b& d & a\\
b& d & a
\end{pmatrix} \in M(3) ~;~
a,b,c, d \in \mathbb{R} \right\}
\)
\( 
E_5 = \left\{  A = \begin{pmatrix}
a& 0 & 0\\
b& 0 & -a\\
b& 0 & a
\end{pmatrix} \in M(3) ~;~
a,b \in \mathbb{R} \right\}
\)

\sphinxAtStartPar
\( 
E_5 = \left\{  A \in M(2) ~;~
A' = -A \right\}
\)

\sphinxAtStartPar
\( 
E_6 = \left\{  A \in M(3) ~;~
A' = A \right\}
\)

\sphinxAtStartPar
tels que \(M(2)\), \(M(3)\) sont resp. les espaces vectoriels des matrices carrées d’ordre 2 et 3.
\begin{enumerate}
\sphinxsetlistlabels{\arabic}{enumi}{enumii}{}{.}%
\item {} 
\sphinxAtStartPar
Parmi ces ensembles, quels sont ceux qui sont des sous\sphinxhyphen{}espaces vectoriels ?

\item {} 
\sphinxAtStartPar
Donner, \sphinxstyleemphasis{en justifiant vos réponses}, une famille génératrice pour chaque sous\sphinxhyphen{}espace vectoriel.

\end{enumerate}
\end{sphinxadmonition}

\begin{sphinxadmonition}{note}{Réponse de l’exercice 3}

\sphinxAtStartPar
On a :
\begin{enumerate}
\sphinxsetlistlabels{\arabic}{enumi}{enumii}{}{.}%
\item {} 
\sphinxAtStartPar
Parmi ces ensembles, quels sont ceux qui sont des sous\sphinxhyphen{}espaces vectoriels ?

\end{enumerate}

\sphinxAtStartPar
A est\sphinxhyphen{}il un sous\sphinxhyphen{}espace vectoriel !?
\end{sphinxadmonition}


\subsection{3. Famille libre}
\label{\detokenize{S_xe9rie1:famille-libre}}
\begin{sphinxadmonition}{note}{Exercice 4}

\sphinxAtStartPar
Les familles de vecteurs suivantes sont\sphinxhyphen{}elles libres ?
\begin{enumerate}
\sphinxsetlistlabels{\arabic}{enumi}{enumii}{}{.}%
\item {} 
\sphinxAtStartPar
\(e_1(0,1,3)\), \(e_2(-1,1,0)\) et \(e_3(-2,0,5)\) dans \(\mathbb{R}^3\).

\item {} 
\sphinxAtStartPar
\(e_1(0,1,1)\), \(e_2(1,1,0)\) et \(e_3(1,1,1)\) dans \(\mathbb{R}^3\).

\item {} 
\sphinxAtStartPar
\(e_1(5,1,-1,0)\), \(e_2(0,-1,3,7)\) et \(e_3(8,1,-1,7)\) dans \(\mathbb{R}^4\).

\end{enumerate}
\end{sphinxadmonition}

\begin{sphinxadmonition}{note}{Réponse de l’exercice 4}

\sphinxAtStartPar
Les familles de vecteurs suivantes sont\sphinxhyphen{}elles libres ?
\end{sphinxadmonition}


\subsection{4. Famille génératrice}
\label{\detokenize{S_xe9rie1:famille-generatrice}}
\begin{sphinxadmonition}{note}{Exercice 5}
\begin{itemize}
\item {} 
\sphinxAtStartPar
La famille de vecteurs \(E = \left\{(1; 1; 0); (1;-1; 0); (1; 0; -1)\right\}\) est\sphinxhyphen{}elle une famille génératrice de \(\mathbb{R}^3\) ? \sphinxstyleemphasis{Justifier}

\end{itemize}
\end{sphinxadmonition}

\begin{sphinxadmonition}{note}{Réponse de l’exercice 5}

\sphinxAtStartPar
\(E = \left\{(1; 1; 0); (1;-1; 0); (1; 0; -1)\right\}\) est\sphinxhyphen{}elle une famille génératrice de \(\mathbb{R}^3\) ?
\end{sphinxadmonition}

\begin{sphinxVerbatim}[commandchars=\\\{\}]

\end{sphinxVerbatim}


\chapter{Série 2 : Application linéaire}
\label{\detokenize{S_xe9rie2:serie-2-application-lineaire}}\label{\detokenize{S_xe9rie2::doc}}






\renewcommand{\indexname}{Index}
\printindex
\end{document}